\chapter{Review of CADe and CADx for CaP}\label{chap:3}
% the code below specifies where the figures are stored
\ifpdf
    \graphicspath{{3_review/figures/}}
\else
    \graphicspath{{3_review/figures/}}
\fi


As previously mentioned in the introduction (see \acs{sec}~\ref{sec:intro:cad}), \acp{cad} are developed to advise and backup radiologists in their tasks of \ac{cap} detection and diagnosis, but not to provide fully automatic decisions \cite{Giger2008}.
\acp{cad} can be divided into two different sub-groups either as \ac{cade}, with the purpose to highlight probable lesions in \ac{mri} images, or \ac{cadx}, which focuses on differentiating malignant from non-malignant tumours \cite{Giger2008}.
Moreover, an intuitive approach, motivated by developing a framework combining detection-diagnosis, is to mix both \ac{cade} and \ac{cadx} by using the output of the former mentioned as a input of the latter named.
Although the outcomes of these two systems should differ, the framework of both \ac{cad} systems is similar.
A general \ac{cad} work-flow is presented n \acs{fig}~\ref{fig:wkfcad}. 
%The \ac{cad} work-flow is presented in \acs{fig}~\ref{fig:wkfcad}.
%%%% Up to here 
\ac{mri} modalities mentioned in \acs{sec}~\ref{subsubsec:mrimrsi} are used as inputs of \ac{cad} for \ac{cap}. It can be noted that \ac{adc} map is not considered as an input since it is a feature derived from the \ac{dw} \ac{mri} images. The images acquired from the different modalities show a large variability between patients: the prostate organ can be located at different positions in images (e.g., patient motion, variation of acquisition plan), and the \ac{si} can be corrupted with noise or artefacts during the acquisition process (eg., magnetic field inhomogeneity, use of endorectal coil). To address these issues, the first stage of \ac{cad} is to pre-process multiparametric \ac{mri} images to reduce noise, remove artefacts and standardize the \ac{si}. Then, it is important to mention that most of the later processes would be only focused on the prostate. Thus, it is necessary to segment the prostate in each \ac{mri}-modality to define it as \iac{roi}. However, data suffers of misalignment due to patient motions or different acquisition plan. Therefore, a registration step is performed so that all the previously segmented \ac{mri} images will be in the same reference frame.

Some studies do not fully apply the methodology depicted in \acs{fig}~\ref{fig:wkfcad}. Details about those can be found in \acs{tab}~\ref{tab:sumpap}. Some studies preferred to work directly with raw data in order to demonstrate the robustness of their approaches to noise or artefacts. In some cases, prostate segmentation is performed manually as well as registration. It is also sometimes assumed that no patient motions occur during the acquisition procedure, removing the need of registering the multiparametric \ac{mri} images.

Once the data are regularized, it becomes possible to extract features and classify the data to obtain the probabilistic maps. We refereed this stage to image classification where \ac{cade} and \ac{cadx} are the main components. 

In \iac{cade} framework, possible lesions will be segmented automatically and further used as input of \ac{cadx}. We also included in \ac{cade} studies, the methods using voxel-based delineation in which the final results will highlight the boundaries of the lesions. On the other hand, manual lesions segmentation are not considered to be part of \iac{cade}. The output of the \ac{cade} is used as input of the \ac{cadx}.

\Ac{cadx} is composed of the processes allowing to distinguish malignant from non-malignant tumours. We divided \ac{cadx} into three different stages. First, salient features are extracted, in an pixel-based or region-based manner, from \ac{mri} images to characterize the lesion. Of course, more discriminative features will be associated with a robust and accurate likelihood cancer map. Frequently, the number of features extracted can be large resulting in redundant or insufficient discriminative features which will negatively affect the performances of the further classification. Therefore, a step consists of selecting the best features or/and reducing the number of dimensions is commonly used. Then, this modified feature vector is finally classified using different pattern recognition approaches.

As pointed out in the introduction, performance of \ac{cap} detection and diagnosis are affected by observer interpretation and limitations \cite{Giger2008,Hambrock2013}. \ac{cad} offers a possible solution in order to reduce this variability. As mentioned in the introduction, the effects of \ac{cad} on the observer performance has been studied \cite{Hambrock2013}, with results showing that \acp{cad} benefit to less-experienced radiologist to perform similarly as experienced radiologist in their tasks \cite{Hambrock2013}. 


\section{Literature classification}\label{sec:chp3:Literature-classification}

\newgeometry{left=1cm,right=1cm,bottom=0.5cm,top=0.3cm}

\begin{table}
\centering
\caption{Overview of the different studies reviewed with their main characteristics. Acronyms: number (\#) - image regularization (Img. Reg.).}
\scriptsize
%\begin{adjustwidth}{-1.5cm}{}
\begin{threeparttable}
\renewcommand{\arraystretch}{1.5}	
	\rowcolors{3}{black!5}{white}	
	\begin{tabular}{|>{\centering\arraybackslash}m{0.7cm}|>{\centering\arraybackslash}m{3.3cm}|>{\centering\arraybackslash}m{1cm}|>{\centering\arraybackslash}m{0.8cm}>{\centering\arraybackslash}m{0.8cm}>{\centering\arraybackslash}m{1cm}>{\centering\arraybackslash}m{1cm}|>{\centering\arraybackslash}m{0.7cm}>{\centering\arraybackslash}m{0.7cm}|>{\centering\arraybackslash}m{0.7cm}>{\centering\arraybackslash}m{0.7cm}|>{\centering\arraybackslash}m{0.7cm}>{\centering\arraybackslash}m{0.7cm}>{\centering\arraybackslash}m{0.7cm}|}\hline
	\hiderowcolors
	\multirow{2}{*}{Index} & \multirow{2}{*}{Study} & \# & \multicolumn{4}{c|}{\ac{mri}-modality} & \multicolumn{2}{c|}{Strength of field} & \multicolumn{2}{c|}{Studied zones} & \multicolumn{3}{c|}{\ac{cad} stages} \\ \cline{4-14}
	 & & patients & \ac{t2w} \ac{mri} & \ac{dce} \ac{mri} & \ac{dw} \ac{mri} & \ac{mrsi} & 1.5 T & 3.0 T & \ac{pz} & \ac{cg} & Img. Reg. & \ac{cade} & \ac{cadx} \\ \hline \hline
	 \showrowcolors 
	 	 $[1]$&\cite{Ampeliotis2007} & 25 & \cmark & \cmark & \xmark & \xmark & \cmark & \xmark & \cmark & \xmark & \mmark & \xmark & \cmark \\
	 	 $[2]$&\cite{Ampeliotis2008} & 25 & \cmark & \cmark & \xmark & \xmark & \cmark & \xmark & \cmark & \xmark & \mmark & \xmark & \cmark \\
	 	 $[3]$&\cite{Antic2013} & 53 & \cmark & \xmark & \cmark & \xmark & \cmark & \xmark & \cmark & \cmark & \xmark  & \xmark & \cmark \\
	 	 $[4]$&\cite{Artan2009} & 10 & \cmark & \cmark & \cmark & \xmark & \cmark & \xmark & \cmark & \xmark  & \xmark & \cmark & \cmark \\
	 	 $[5]$&\cite{Artan2010} & 21 & \cmark & \cmark & \cmark & \xmark & \cmark & \xmark & \cmark & \xmark & \mmark & \cmark & \cmark \\
	 	 $[6]$&\cite{Chan2003} & 15 & \cmark & \xmark & \cmark & \xmark & \cmark & \xmark & \cmark & \xmark & \xmark & \xmark & \cmark \\
	 	 $[7]$&\cite{Giannini2013} & 10 & \cmark & \cmark & \cmark & \xmark & \cmark & \xmark & \cmark & \xmark & \cmark & \cmark & \cmark \\
	 	 $[8]$&\cite{Kelm2007} & 24 & \xmark & \xmark & \xmark & \cmark & \cmark & \xmark & \cmark & \cmark & \mmark & \cmark & \cmark \\
	 	 $[9]$&\cite{Langer2009} & 25 & \cmark & \cmark & \cmark & \xmark & \cmark & \xmark & \cmark & \xmark & \mmark & \xmark & \cmark \\
	 	 $[10]$&\cite{Litjens2011} & 188 & \cmark & \cmark & \cmark & \xmark & \xmark & \cmark & \cmark & \xmark & \mmark & \cmark & \cmark \\
	 	 $[11]$&\cite{Litjens2012} & 288 & \cmark & \cmark & \cmark & \xmark & \xmark & \cmark & \cmark & \cmark & \mmark & \cmark & \cmark \\
	 	 $[12]$&\cite{Liu2009} & 11 & \cmark & \cmark & \cmark & \xmark & \cmark & \xmark & \cmark & \xmark & \mmark & \cmark & \cmark \\
	 	 $[13]$&\cite{Liu2013} & 54 & \cmark & \cmark & \cmark & \xmark & \xmark & \cmark & \cmark & \cmark & \mmark & \xmark & \cmark \\
	 	 $[14]$&\cite{Lopes2011} & 27 & \cmark & \xmark & \xmark & \xmark & \cmark & \xmark & \cmark & \xmark & \mmark & \cmark & \cmark \\
	 	 $[15]$&\cite{Lv2009} & 55 & \cmark & \xmark & \xmark & \xmark & \cmark & \xmark & \cmark & \xmark & \mmark & \xmark & \cmark \\
	 	 $[16]$&\cite{Matulewicz2013} & 18 & \xmark & \xmark & \xmark & \cmark & \xmark & \cmark & \cmark & \cmark & \xmark & \cmark & \cmark \\ 
	 	 $[17]$&\cite{Mazzetti2011} & 10 & \xmark & \cmark & \xmark & \xmark & \cmark & \xmark & \cmark & \xmark & \mmark & \cmark & \cmark \\
	 	 $[18]$&\cite{Niaf2011} & 23 & \cmark & \cmark & \cmark & \xmark & \cmark & \xmark & \cmark & \xmark & \mmark & \xmark & \cmark \\
	 	 $[19]$&\cite{Niaf2012} & 30 & \cmark & \cmark & \cmark & \xmark & \cmark & \xmark & \cmark & \xmark & \mmark & \xmark & \cmark \\
	 	 $[20]$&\cite{Ozer2009} & 20 & \cmark & \cmark & \cmark & \xmark & \cmark & \xmark & \cmark & \xmark & \mmark & \cmark & \cmark \\
	 	 $[21]$&\cite{Ozer2010} & 20 & \cmark & \cmark & \cmark & \xmark & \cmark & \xmark & \cmark & \xmark & \mmark & \cmark & \cmark \\
	 	 $[22]$&\cite{Parfait2012} & 22 & \xmark & \xmark & \xmark & \cmark & \xmark & \cmark & \cmark & \cmark & \mmark & \cmark & \cmark \\
	 	 $[23]$&\cite{Peng2013} & 48 & \cmark & \cmark & \cmark & \xmark & \xmark & \cmark & \cmark & \cmark & \xmark & \xmark & \cmark \\
	 	 $[24]$&\cite{Puech2009} & 100 & \xmark & \cmark & \xmark & \xmark & \cmark & \xmark & \cmark & \cmark & \xmark & \xmark & \cmark \\
	 	 $[25]$&\cite{Sung2011} & 42 & \xmark & \cmark & \xmark & \xmark & \xmark & \cmark & \cmark & \cmark & \xmark & \cmark & \cmark \\
	 	 $[26]$&\cite{Tiwari2007} & 14 & \xmark & \xmark & \xmark & \cmark & \cmark & \xmark & \cmark & \cmark & \mmark & \cmark & \cmark \\
	 	 $[27]$&\cite{Tiwari2008} & 18 & \xmark & \xmark & \xmark & \cmark & \cmark & \xmark & \cmark & \cmark & \mmark & \cmark & \cmark \\
	 	 $[28]$&\cite{Tiwari2009} & 18 & \xmark & \xmark & \xmark & \cmark & \cmark & \xmark & \cmark & \cmark & \mmark & \cmark & \cmark \\
	 	 $[29]$&\cite{Tiwari2009a} & 15 & \cmark & \xmark & \xmark & \cmark & \cmark & \xmark & \cmark & \cmark & \mmark & \cmark & \cmark \\
	 	 $[30]$&\cite{Tiwari2010} & 19 & \cmark & \xmark & \xmark & \cmark & \cmark & \xmark & \cmark & \cmark & \mmark & \cmark & \cmark \\
	 	 $[31]$&\cite{Tiwari2012} & 36 & \cmark & \xmark & \xmark & \cmark & \cmark & \xmark & \cmark & \cmark & \xmark & \cmark & \cmark \\
	 	 $[32]$&\cite{Tiwari2013} & 29 & \cmark & \xmark & \xmark & \cmark & \cmark & \xmark & \cmark & \cmark & \mmark & \cmark & \cmark \\
	 	 $[33]$&\cite{Viswanath2008} & 16 & \cmark & \xmark & \xmark & \cmark & \cmark & \xmark & \cmark & \cmark & \xmark & \cmark & \cmark \\
	 	 $[34]$&\cite{Viswanath2008a} & 6 & \cmark & \cmark & \xmark & \xmark & \xmark & \cmark & \cmark & \cmark & \mmark & \cmark & \cmark \\
	 	 $[35]$&\cite{Viswanath2009} & 6 & \cmark & \cmark & \xmark & \xmark & \xmark & \cmark & \cmark & \cmark & \cmark & \cmark & \cmark \\
	 	 $[36]$&\cite{Viswanath2011} & 12 & \cmark & \cmark & \cmark & \xmark & \xmark & \cmark & \cmark & \cmark & \mmark & \cmark & \cmark \\
	 	 $[37]$&\cite{Viswanath2012} & 22 & \cmark & \xmark & \xmark & \xmark & \xmark & \cmark & \cmark & \cmark & \cmark & \cmark & \cmark \\
	 	 $[38]$&\cite{Vos2008} & 29 & \cmark & \cmark & \xmark & \xmark & \cmark & \xmark & \cmark & \xmark & \mmark & \xmark & \cmark \\
	 	 $[39]$&\cite{Vos2008a} & 29 & \xmark & \cmark & \xmark & \xmark & \cmark & \xmark & \cmark & \xmark & \mmark & \xmark & \cmark \\
	 	 $[40]$&\cite{Vos2010} & 29 & \cmark & \cmark & \xmark & \xmark & \cmark & \xmark & \cmark & \xmark & \mmark & \xmark & \cmark \\
	 	 $[41]$&\cite{Vos2012} & NA & \cmark & \cmark & \cmark & \xmark & \xmark & \cmark & \cmark & \xmark & \mmark & \cmark & \cmark \\
	 	 \hline
	\end{tabular}
	\begin{tablenotes}
      \tiny
      \item Notes:
      \item {\xmark}: not used or not implemented.
      \item {\mmark}: partially implemented.
      \item {\cmark}: used or implemented.
    \end{tablenotes}
\end{threeparttable}
%\end{adjustwidth}
\label{tab:sumpap}
\end{table}

\restoregeometry

The \ac{cad} review is organized using the methodology presented in \acs{fig}~\ref{fig:wkfcad}. Methods embedded in the image regularization framework are presented before to focus on the image classification framework, the later being divided into \ac{cade} and \ac{cadx}. \Acl{tab}~\ref{tab:sumpap} summarizes the different \ac{cad} studies reviewed in this paper. Characteristics related to \ac{mri} acquisition as well as \ac{cad} strategies are reported. Only methods used in \ac{cad} system are discussed.


